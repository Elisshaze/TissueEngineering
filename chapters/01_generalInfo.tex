\chapter{General course information}
  \section{Textbooks}
    \begin{itemize}
      \item Scaffold for Tissue Engineering: Biological Design, Materials, and Fabrication
      \item Introduction to Tissue material interactions (PDF)
      \item Selected chapters from : Principles of Regenerative Medicine (prof. Atala) (PDF)
      \item Selected chapters from: Adult wound healing (Yannis), and Extreme Tissue Engineering
      \item Selected scientific papers
    \end{itemize}

  \section{Assessment}
    \begin{itemize}
      \item \textit{TODO Still trying to define exactly}
    \end{itemize}

  \section{Topics}
    \begin{itemize}
      \item Concept of therapeutic device, transplant or implant?
      \item From tissue substitution to tissue regeneration: Introduction to tissue engineering
      \item Biocompatibility: concept evolution, mechanism, new approaches
      \item Structure, function, of ECM, role in cell activity, Cell/ECM interactions
      \item Wound healing: repair, regeneration, scar tissue formation
      \item Foreign body reaction: immuno, inflammatory, blood coagulation, complement system ECM as a model for scaffold design: engineering biomimetic scaffolds
      \item Polymers, biopolymers, hydrogels, fabrication methods
      \item Material/biological system interactions
      \item Strategies to control scaffold vascularization
      \item Strategies in TE: from top down to bottom up approach
      \item Organ printing and cell encapsulation
      \item TE applied to 3D in vitro models and lab-on-chip: drug screen, cancer studies, personalized medicine.
      \item In vitro-in vivo evaluations: bioreactors
      \item Lectures in lab: practical demonstration on scaffold fabrication
    \end{itemize}