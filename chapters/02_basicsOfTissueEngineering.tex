\chapter{Basics of tissue engineering}

\section{Definition of tissue engineering}

  \subsection{First definition}
  At first, \textbf{tissue engineering} was defined as \textit{"A combination of principles and methods of life sciences with that of engineering, to develop materials and methods to repair damaged or diseased tissues, and to create entire tissue/organs replacements"} (1980s).
  This definition was soon outdated since it did not consider a fundamental difference:

  \begin{multicols}{2}
    \begin{itemize}
      \item \textbf{Repair} is the act of closing the wound, restoring the macroscopic structure; the tissue  formed during repair is often different from the starting tissue (scar tissue for instance), and thus the different properties undermine the function of the repaired area due to the different properties.
      \item \textbf{Regeneration} is also a way of closing a wound, but the structure is restored using cells of the same type as the starting one, therefore maintaining the functionality of the area.
    \end{itemize}
  \end{multicols}

  Tissue engineering in fact aims at regenerating the structure and functionality of the district, not merely closing the wound.

  \subsection{Regeneration-centric definition}
  A following definition puts the focus on the regeneration aspect, underlining the necessity to understand the mechanism guiding regeneration: \textit{"The applications of principles and methods of engineering and life sciences, to obtain a fundamental understanding of structural and functional relationships in novel and pathological mammalian tissues, and the development of biological substitutes to restore, maintain or improve tissue function"}(late 1980s).

  \subsection{Current definition}
  Nowadays, tissue engineering could be defined as \textit{"a biomedical engineering discipline that uses a combination of cells, engineering, materials, methods and suitable biochemical and physicochemical factors to restore, maintain, improve or replace different types of biological tissues."}
  Tissue engineering holds promise of producing healthy organs for transplant by using patient cells (or immuno compatible cells). Tissue engineering can be combined with gene therapy, therefore including the correction of incurable genetic defects (such as sickle cell anemia).


\section{History of tissue engineering}
Tissue engineering is born from the concept that interaction between cells and extracellular matrix is important for understanding the structural and functional relationship of these components.
The first experiment on tissue engineering was conducted by W.T. Green, who tried to regenerate bone using chondrocytes (since in the physiological process, bone is obtained by calcification of cartilage); he managed to obtain bone formation in nude mice, thus he concluded that with the advent of innovative biocompatible materials it would be possible to generate new tissue by seeding viable cells onto appropriately configured scaffolds.
Later on, Langer and Vacanti created artificial scaffolds for cell-delivery, rather than using natural derived scaffolds which are difficult to replicate. The use of artificial matrices specifically designed for the system allowed to obtain reproducibility and high-quality.
(\textit{Skipped other experiments and history of tissue engineering})

\section{The need for tissue engineering}
Tissue/organ transplant is a heavily limited solution for tissue/organ failure; some of the main limitations being:

\begin{multicols}{2}
  \begin{itemize}
    \item Donor-recipient compatibility: it is almost impossible to find a fully compatible donor (all major histocompatibility complexes matching with the recipient) and the use of non-fully compatible organs/tissues requires the recipient to undergo immunosuppresive therapy (generally chronically) to avoid rejection of the transplant.
    \item Rejection risk: rejection can occur regardless of compatibility and immunosuppressive therapy, therefore this risk can never be avoided completely.
    \item Organ/tissue scarcity: even not considering compatibility, the amount of organs/tissues that can be donated is very scarce, since most of them come from car accident victims or relatives.
  \end{itemize}
\end{multicols}

For this reasons implanting artificial devices has grown more popular since:

\begin{multicols}{2}
  \begin{itemize}
    \item They are ready to use.
    \item They immediately restore the function of the organ/tissue.
    \item They can be personalyzed.
    \item They do not cause rejection.
  \end{itemize}
\end{multicols}

Still, implants present many limitation:

\begin{multicols}{2}
  \begin{itemize}
    \item They require invasive surgery.
    \item They can cause foreign body reaction.
    \item They cannot replace completely the functions of the organ/tissue (limited performance).
    \item They have limited duration.
  \end{itemize}
\end{multicols}

  \subsection{Examples of implant devices}
  Some examples of implant devices are:

  \begin{multicols}{2}
    \begin{itemize}
      \item \textbf{Hip joint prosthesis}: made of metallic alloys (mostly based on titanium) and ceramic (for the joint socket).
        It immediately restores the function of the joint but it requires very invasive procedures (long segment inside the femur) and overtime it sticks to the bone, making it really hard to substitute it.
        This last point is of little relevance if we consider older people, but it is a big problem for younger ones.
      \item \textbf{Vascular stent}: a stent is a cylindrical tool that is used in angioplastic procedures, meaning it prevents the stenosis (blockage) of (usually coronary) arteries.
        A flexible probe mounted with a stent, a balloon and some way to visualize the probe from outside the body (via ecography for example) is inserted in the femural artery; then the probe is navigated to the damaged region and the balloon inflates positioning the stent.
        This allows to restore blood flow to the miocardium without open heart surgery, even in an emergency setting such as in case of heart attack.
        The main problem is that the artery keeps contracting and expanding, therefore the stent can damage the vessel causing necrosis, proliferation of fibrotic tissue, inflammation; one way to reduce the problem is to use polymers rather than metallic alloys, since they are more flexible, but they are also less durable.
        Moreover stents are in contact with the blood and thus provide an abnormal surface that can start platelette aggregation and coagulation, creating blood clots; the use of slow release anticoagulant drugs stents helps with that aspect.
      \item \textbf{Artificial heart}: heart shaped device with valves.
        This device does not work as a pump since it cannot contract like miocardial fibres, therefore it require some form of auxiliary external pump.
        Just like with stents you can have abnormal coagulation on the surface of the device or due to turbolences in the blood flow.
      \item \textbf{Bone scaffolds}: they are three-dimensional biomaterial structures used for bone defect reconstruction.
        An ideal scaffold should have features such as improving cell adhesion, proliferation, osteogenic differentiation, vascularization, host integration and, where necessary, load bearing (drugs for instance).
        These design parameters should lead to specific scaffold properties, which include biocompatibility, porosity, micro and nano-scale structure, degradation rate, mechanical strength, and growth factor delivery, all of which dictate the biomaterial to be used or developed (\textit{further explaination later on}).
    \end{itemize}
  \end{multicols}

\section{Tissue engineering paradigma}
Since artificial devices do not replace all the functions of a lost organ or tissue and often fail in the long term, a new approach stems from two considerations:

\begin{multicols}{2}
  \begin{itemize}
    \item Living tissues and organs can be routinely assembled and reliably integrated to the body to restore, replace or enhance tissue and organ functions.
    \item Biomaterials can interact with living tissue and influence cell function and response.
  \end{itemize}
\end{multicols}

This approach is in fact tissue engineering which is, using yet another definition for it, \textit{"creation of a new tissue for therapeutic reconstruction of the human body, by the deliberate and controlled stimulation of selected target cells, through a systematic combination of molecular and mechanical signals"}.
Basically you are not creating ex novo a tissue/organ to implant, but you are inducing some biological mechanisms that helps the body heal itself; this idea is more generally represented by the term \textbf{regenerative medicine}, which includes not only tissue engineering but also cell therapy and gene therapy.
The main actors involved in this process are:

\begin{multicols}{2}
  \begin{itemize}
    \item \textbf{Cells}: generally stem cells derived from the patient (to avoid rejection).
    \item \textbf{Scaffold}: which is the structure used to induce and guide the growth of the cells; tissue engineering focusses mostly on material, surface and properties of this component.
    \item \textbf{Time}: required in order to induce and obtain regeneration in unnatural conditions.
  \end{itemize}
\end{multicols}

  \subsection{Scaffold characteristics}
  The main characteristics of a scaffold that must be considered are:

  \begin{multicols}{2}
    \begin{itemize}
      \item \textbf{Mechanical properties}: ability to withstand mechanical stress (elasticity, compressibility...).
      \item \textbf{Morphology}: shape, size, structure...
      \item \textbf{Physical properties}: behaviour when temperature, pH and other aspects of the environment change.
      \item \textbf{Histology}: type of tissue it has to replace.
      \item \textbf{Porosity}: permeability of the structure to different elements; it must allow the cells of interest to grow and penetrate into the structure while keeping outside unwanted cell types.
      \item \textbf{Water content}: the structure must contain enough water to allow nutrients to reach all the cells.
      \item \textbf{Surface}: the surface of the scaffold must be functionalized with molecules that are recognized by surface receptors of the target cells, thus inducing gene activation and some form of response (adhesion, expression upregulation/downregulation...).
    \end{itemize}
  \end{multicols}

  From these premises we get the so called \textbf{tissue engineering paradigma}, which is the basic flowchart that most tissue engineering procedures follow.
  In general, the main steps are:

  \begin{multicols}{2}
    \begin{itemize}
      \item \textbf{Collecting and isolating host cells of interest}: the type of cells needed for the procedure depends on the damage site; according to the cells needed (generally stem cells, sometimes primary cells) a biopsy is performed on an adequate tissue (blood, skin...).
        Moreover, since tissue are generally heterogeneous, some steps are required to isolate the cells of interest from the others.
        The fact that donor and recipient coincide, there are no compatibility issues.
      \item \textbf{Seeding cells on a scaffold}: the cells are then seeded on a scaffold made of some biomaterial which is biocompatible for the application at hand (\textit{biocompatibility characteristics are discussed later on}).
        This scaffold must provide adequate conditions for cell growth and proliferation, such as the presence of growth factors and cytokines.
      \item \textbf{Cell stimulation in a bioreactor}: The seeded scaffold is then placed into a static or dinamic bioreactor which induces and stimulate cell growth and proliferation.
        This bioreactor must be able to provide all the stimuli needed for cell differentiation and organization according to the desired final result (this may include mechanical stress for instance, needed for miocardial differentiation, or type of surface, since the differentiation of chondrocytes depends on the form they assume due to adhesion).
      \item \textbf{Re-implantation}: the fully prepared and populated construct is then implanted into the damaged area, where it will integrate itself with the surrounding tissues.
    \end{itemize}
  \end{multicols}

  This approach has some major downsides, namely:

  \begin{multicols}{2}
    \begin{itemize}
      \item It is very \textbf{labour intensive} and \textbf{time consuming} since it takes time for the cells to proliferate and populate the scaffold, moreover the growth conditions depend on the cell type of interest (which can be difficult to define, since the whole physiological environment must be taken into account, therefore angiogenesis, immune system, vacularization, lymphatic system and much more).
      \item Given the production time, this approach is \textbf{not ready to use}, therefore it cannot be used in an emergency setting.
      \item The amount of time and labour needed for the production implies \textbf{high cost}.
    \end{itemize}
  \end{multicols}

  In some cases, it is possible to simplify the procedure by implanting the scaffold immediately after it was populated with cells in the damaged area; this is called \textbf{in-situ regeneration}.
  In situ regeneration requires way less preparation time (almost ready to use) and uses the body of the recipient as a bioreactor, therefore reducing labour (you already have all the machinery required), need for bioreactor setup and overall costs.
  Notice that model animals cannot be used as bioreactors, since the mechanical properties of the tissues are not comparable (vertebrae of pigs and sheeps are built to support different weights compared to the human ones), the use of primates is strictly regulated by law and no animal is perfectly compatible with humans (therefore you risk rejection).
  The in-situ regeneration approach is not always applicable and just like the slower version it has a lot of room for improvement, for instance the development of more functional and durable polymers to use as scaffolds.
  Another problem for both strategies is the need of starting material, generally stem cells: the patient may not have enough excess tissue for an autologous transplant for instance.
  One way to mitigate this problem would be to save part of the umbilical cord to extract staminal cells, but in Italy this is forbidden by law (\textit{as of writing the notes}).

\section{Tissue engineering approaches}
There are two different \textbf{approaches} for tissue engineering:

\begin{multicols}{2}
  \begin{itemize}
    \item \textbf{Top-down approach} (traditional approach): cells are harvested from the donor, cultured and modified if needed. They are then seeded on a porous scaffold that during cell proliferation is slowly degraded by the cells and replaced by extra cellular matrix (ECM). The engineered tissue is then implanted into the patient. The main advantage of this procedure is that it is possible to produce mechanical stress, which is needed for the differentiation on certain types of cells.
    \item \textbf{Bottom-up approach} (modular approach): Some fundamental elements, such as cell sheets, cell aggregates, cell laden modules and bioink (3D printer ink containing cells) are used to construct a 3D module assembly, which can then be implanted into the patient. This modular building process allows to create very complex structures, with gradients and without a scaffold. Since a temporary gelatinous matrix is used, the module assembly lacks the rigidity provided by a scaffold, therefore it does not easily maintain the mechanical stress. Furthermore, when 3D printing, many other aspects have to be taken into account, like the permeability of the matrix to nutrients, the sensibility of the cells to the stress due to the extrusion from the needle and others.
  \end{itemize}
\end{multicols}
