\textbf{Tissue engineering} is a biomedical engineering discipline that uses a combination of cells, engineering, materials, methods and suitable biochemical and physicochemical factors to restore, maintain, improve or replace different types of biological tissues.


Tissue/organ transplant is a heavily limited solution for tissue/organ failure; some of the main limitations being:
\begin{itemize}
  \item Donor-recipient compatibility: it is almost impossible to find a fully compatible donor (all major histocompatibility complexes matching with the recipient) and the use of non-fully compatible organs/tissues requires the recipient to undergo immunosuppresive therapy (generally chronically) to avoid rejection of the transplant. 
  \item Rejection risk: rejection can occur regardless of compatibility and immunosuppressive therapy, therefore this risk can never be avoided completely.
  \item Organ/tissue scarcity: even not considering compatibility, the amount of organs/tissues that can be donated is very scarce, since most of them come from car accident victims or relatives. 
\end{itemize}
For this reasons implanting artificial devices has grown more popular since:
\begin{itemize}
  \item They are ready to use
  \item They immediately restore the function of the organ/tissue
  \item They can be personalyzed
  \item They do not cause rejection
\end{itemize}
Still, implants present many limitation:
\begin{itemize}
  \item They require invasive surgery
  \item They can cause foreign body reaction 
  \item They cannot replace completely the functions of the organ/tissue (limited performance)
  \item They have limited duration
\end{itemize}

Examples of implant devices:
\begin{itemize}
  \item \textbf{Hip joint prosthesis}: made of metallic alloys (mostly based on titanium) and ceramic (for the joint socket). It immediately restores the function of the joint but it requires very invasive procedures (long segment inside the femur) and overtime it sticks to the bone, making it really hard to substitute it. This last point is of little relevance if we consider older people, but it is a big problem for younger ones.
  \item \textbf{Vascular stent}: a stent is a cylindrical tool that is used in angioplastic procedures, meaning it prevents the stenosis (blockage) of (usually coronary) arteries. A flexible probe mounted with a stent, a balloon and some way to visualize the probe from outside the body (via ecography for example) is inserted in the femural artery; then the probe is navigated to the damaged region and the balloon inflates positioning the stent. This allows to restore blood flow to the miocardium without open heart surgery, even in an emergency setting such as in case of heart attack. The main problem is that the artery keeps contracting and expanding, therefore the stent can damage the vessel causing necrosis, proliferation of fibrotic tissue, inflammation; one way to reduce the problem is to use polymers rather than metallic alloys, since they are more flexible, but they are also less durable. Moreover stents are in contact with the blood and thus provide an abnormal surface that can start platelette aggregation and coagulation, creating blood clots; the use of slow release anticoagulant drugs stents helps with that aspect. 
  \item \textbf{Artificial heart}: heart shaped device with valves. This device does not work as a pump since it cannot contract like miocardial fibres, therefore it require some form of auxiliary external pump. Just like with stents you can have abnormal coagulation on the surface of the device or due to turbolences in the blood flow.
  \item \textbf{Bone scaffolds}: they are three-dimensional biomaterial structures used for bone defect reconstruction. An ideal scaffold should have features such as improving cell adhesion, proliferation, osteogenic differentiation, vascularization, host integration and, where necessary, load bearing (drugs for instance). These design parameters should lead to specific scaffold properties, which include biocompatibility, porosity, micro and nano-scale structure, degradation rate, mechanical strength, and growth factor delivery, all of which dictate the biomaterial to be used or developed (\textit{further explaination later on}.  % TODO add link on biocompatibility section? 
\end{itemize}


Since artificial devices do not replace all the functions of a lost organ or tissue and often fail in the
long term, a new approach stems from two considerations:
\begin{itemize}
  \item Living tissues and organs can be routinely assembled and reliably integrated to the body to restore, replace or enhance tissue and organ functions.
  \item Biomaterials can interact with living tissue and influence cell function and response.
\end{itemize}
From these premises we get the so called \textbf{tisse engineering paradigma}, which is the basic flowchart that most tissue engineering procedures follow. In general, the main steps are:
\begin{itemize}
  \item \textbf{Collecting and isolating host cells of interest}: the type of cells needed for the procedure depends on the damage site; according to the cells needed (generally stem cells, sometimes primary cells) a biopsy is performed on an adequate tissue (blood, skin...). Moreover, since tissue are generally heterogeneous, some steps are required to isolate the cells of interest from the others. The fact that donor and recipient coincide, there are no compatibility issues.
  \item \textbf{Seeding cells on a scaffold}: the cells are then seeded on a scaffold made of some biomaterial which is biocompatible for the application at hand (\textit{biocompatibility characteristics are discussed later on}). This scaffold must provide adequate conditions for cell growth and proliferation, such as the presence of growth factors and cytokines.
  %TODO Add link to biocompatibility reference?
  \item \textbf{Cell stimulation in a bioreactor}: The seeded scaffold is then placed into a static or dinamic bioreactor which induces and stimulate cell growth and proliferation. This bioreactor must be able to provide all the stimuli needed for cell differentiation and organization according to the desired final result (this may include mechanical stress for instance).
  \item \textbf{Re-implantation}: the fully prepared and populated construct is then implanted into the damaged area, where it will integrate itself with the surrounding tissues.
\end{itemize}
This approach has some major downsides, namely:
\begin{itemize}
  \item It is very \textbf{labour intensive} and \textbf{time consuming} since it takes time for the cells to proliferate and populate the scaffold, moreover the growth conditions depend on the cell type of interest (which can be difficult to define, since the while physiological environment must be taken into account).
  \item Given the production time, this approach is \textbf{not ready to use}, therefore it cannot be used in an emergency setting.
  \item The amount of time and labour needed for the production imply \textbf{high cost}.
\end{itemize}
In some cases, it is possible to simplify the procedure by implanting the scaffold immediately after it was populated with cells in the damaged area; this is called \textbf{in-situ regeneration}. In situ regeneration requires way less preparation time (almost ready to use) and uses the body of the recipient as a bioreactor, therefore reducing labour, need for bioreactor setup and overall costs. This approach is not always applicable and just like the slower version it has a lot of room for improvement, for instance the development of more functional and durable polymers to use as scaffolds.
