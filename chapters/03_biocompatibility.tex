\chapter{Biocompatibility}
Biocompatibility is essential - if not the most important - aspect to take into conisderation for the specification of the medical device. 
\\
Let's suppose we want ot build a chair. Before building it we must design it in each of its components. We need to figure out how long it is going to be used, where, from whom. The same approach applies to scaffolds and biological devices. A scaffold to be useful should activate specific cellular functions, and each tissue has different chemical and mechanical properties, therefore the scaffold should be designed for the specific region of the body in which it will implanted, taking into consideration the type of cells populating it, the kind of injury, etc.
\\
Back to the chair analogy: for example, which cells will use the device? Cells are the building blocks of biology and can read the information coming from the external enviroment. Given external signaling molecules, specific gene expression is activated to regulate pathways. The enviroment reaches the cell troguh chemical signal, but also through mechanical stimuli. A note on this: scaffold are not \textit{self-sufficient} entities. They are (often) bioactive, but still neeed to to collaborte and make the best out of the enviroment. The ECM composition and the its biocompatibility with the device is as important as the cell-device relationship. 
\\
The biological outcome that will be obtained then depends on a list of parameters:
\begin{description}
\item[Porosity]
\item[Mechanical properties]
\item[Surface]
\item[Antibiotic/antiviral]
\item[Surface topography]
\end{description}
porosity (porose for cells, cells colonization in 3D, porse should be big anough for cells to migrate into), mehcanical properties, surface modification (before modificarion, e.g. adehesion), include drug release (antibiotics, antiviral...), surface topografy (smooth, rough,). Take home lesson: many parameters that can have many differrent values. 